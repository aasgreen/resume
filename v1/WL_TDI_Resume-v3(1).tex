
\documentclass[11.5pt,a4paper]{article}
\usepackage{scrextend}
\usepackage[margin=0.75in]{geometry}
\usepackage[ngerman]{babel}
\usepackage{enumitem}
\usepackage{cleveref}[2012/02/15]
%\usepackage[hidelinks]{hyperref}
\usepackage{hyperref}
%\usepackage{nohyperref}
%\usepackage[hidelinks]{url}
\renewcommand{\baselinestretch}{1.10} 
\setlist[itemize]{noitemsep, topsep=0pt}
\usepackage{titling}
\usepackage{sectsty}
\usepackage{amsmath}

\usepackage{array}
\newcolumntype{L}[1]{>{\raggedright\let\newline\\\arraybackslash\hspace{0pt}}m{#1}}
\newcolumntype{C}[1]{>{\centering\let\newline\\\arraybackslash\hspace{0pt}}m{#1}}
\newcolumntype{R}[1]{>{\raggedleft\let\newline\\\arraybackslash\hspace{0pt}}m{#1}}

\sectionfont{\fontsize{12}{15}\selectfont}

\pagenumbering{gobble}

\setlength{\droptitle}{-0.75In}

\renewcommand{\thefootnote}{\fnsymbol{footnote}}

%\title{\textbf{William E. Lewis}}
%\date{}
%\setlength{\droptitle}{-6.5em}
\begin{document}
%\maketitle
%\vspace{-6.7em}
\hspace*{-\parindent}%
\begin{minipage}{0.75\textwidth}

\begin{flushleft}
\noindent {\large \textbf{William E. Lewis}}\\
%LinkedIn: \url{https://www.linkedin.com/in/william-lewis-a2a106a2/}
Email: willeslewis@gmail.com\\
%\noindent Website: https://sites.google.com/colorado.edu/william-lewis
\end{flushleft}
\end{minipage}%
\hfill \hspace{1em}
\begin{minipage}{0.5\textwidth}
\begin{flushleft}
Phone: 501-680-1310
\vspace{-1.25em}
 \end{flushleft}
\end{minipage} \\
\vspace{-1.65em}

\section*{Key Skills and Highlights}
\vspace{-1.75em}
\noindent\rule{\textwidth}{0.4pt}\\
\vspace{-0.7em}
\phantom{.}\\
\begin{tabular}{L{2.5cm}|p{13.75cm}}
 \noindent \hspace{-0.15cm}\textbf{Programming}& \textit{Python} (pandas, matplotlib, bokeh, beautifulsoup, scikit-learn, tensorflow, keras), Spark, SQL, and Git. Prior experience with \textit{MATLAB}, \textit{R}, \textit{Mathematica} and \textit{C$++$}\\\multicolumn{2}{c}{} \\
 \noindent \hspace{-0.15cm}\textbf{Data Science} & Web scraping, cleaning data, exploratory analysis, data transformation (standard-normal scaling, t-SNE, SVD, PCA),  image/time-series analysis, data visualization, dashboard maintenance, statistical inference (A/B, ANOVA), regression and classification models (k-means, GLMs, boosted trees), deep learning (ANN, CNN, RNN architectures) \\\multicolumn{2}{c}{}\\
\noindent \hspace{-0.15cm}\textbf{Mathematics} &  Bayesian statistics, matrix factorization, eigenvector/eigenvalue problems, convolution, fast Fourier transforms, short-time Fourier transforms, Monte-Carlo\\\multicolumn{2}{c}{} \\
\noindent \hspace{-0.15cm}\textbf{Achievements}& National Barry M. Goldwater Scholarship (2011)\\
& 5 peer-reviewed publications in 3 distinct subfields\\
& 5 oral presentations: 3 national conferences, 1 international conference, 1 colloquium\\\multicolumn{2}{c}{}\\
\noindent \hspace{-0.15cm}\textbf{Other} & Experience working within the Agile framework Scrum\\
&  Code review and unit tests for python and SQL code\\
& Individual and collaborative research with up to 5 team members\\
& Mentorship of 2 undergraduate research students \\\multicolumn{2}{c}{}\\
\end{tabular}

\vspace{-2.25em}

\section*{Education}
\vspace{-1.85em}
\noindent\rule{\textwidth}{0.4pt}\\
\vspace{-0.75em}
\phantom{.}\\
\phantom{.} \textbf{Ph.D. Physics} University of Colorado \dotfill May 2018\\
\indent - \textit{Strongly Interacting Fermi Gases: Hydrodynamics and Beyond}\\
\vspace{-0.75em}
\phantom{.}\\
\phantom{.} \textbf{Data Science Fellow} The Data Incubator \dotfill February 2018\\
\indent - \textit{Churn Prediction in Music Streaming Subscriptions}\\
\vspace{-0.75em}
\phantom{.}\\
\phantom{..}\textbf{M.S. Physics} University of Colorado \dotfill August 2015\\
\vspace{-0.75em}
\phantom{.}\\
\phantom{..}\textbf{B.S. Mathematics, Physics \textit{Summa Cum Laude}} University of Arkansas \dotfill May 2012\\
\indent - \textit{Polarization and propagation characteristics of cylindrical vector beams}\\

\vspace{-2.25em}


\section*{Work Experience}
\vspace{-1.85em}
\noindent\rule{\textwidth}{0.4pt}
\vspace{-0.75em}
\phantom{.}\\
\phantom{.} \textbf{Data Analyst at Jana Mobile} \dotfill March 2018 - Present \\
\indent - Identified assumptions of A/B tests were too restrictive thereby discovering $\mathcal{O}(\$1000)$ per day savings\\
\indent - Utilized terabytes of high-dimensional data in analyses and feature engineering for predictive models\\
\indent - Prototyped a gradient boosted tree model to predict if online purchases would be returned\\

\vspace{-2.25em}


\section*{Research Vignettes}
\vspace{-1.85em}
\noindent\rule{\textwidth}{0.4pt}
\vspace{-0.75em}
\phantom{.}\\
\phantom{.} \textbf{Fluid dynamics in quantum gases} \dotfill May 2015 - December 2017 \\
\indent - Utilized MLE in \textit{Python} to fit models to fluid simulation data using AIC for model selection\\
\indent - Reduced data output size from $\mathcal{O}($1GB$)$ to $\mathcal{O}($10kB$)$ in \textit{C++} Lattice Boltzmann simulations \\
\indent - Worked with analytical and numerical methods of fluid dynamics \\
\indent - Made experimentally verifiable predictions for novel dynamics \\
\vspace{-0.75em}
\phantom{.}\\
\phantom{.} \textbf{Nano-imaging and spectroscopy} \dotfill July 2012 - May 2015\\
\indent - Applied WMA to characterize statistical properties of laser fluctuations in \textit{MATLAB} \\
\indent - Conducted short-time Fourier analysis of time-series data using \textit{MATLAB}\\
\indent - Fit regression models to experimental data in \textit{MATLAB}\\
%\vspace{-2.25em}
\vspace{-0.75em}
\phantom{.}\\
\phantom{.} \textbf{Polarization Structure of Laser Light} \dotfill January 2010 - May 2012\\
\indent - Engineered features to intuitively describe properties of a thirteen-dimensional data set\\
\indent - Related theoretical results to experimentally measurable quantities (e.g. laser power)
%\vspace{-1.0em}
\section*{Additional Activities}
\vspace{-1.85em}
\noindent\rule{\textwidth}{0.4pt}
\vspace{-0.9em}
\phantom{.}\\
\noindent
\phantom{.} \textbf{Deep learning specialization} \dotfill October 2018\\
\indent - Deeplearning.ai Coursera 5 course specialization with weekly python programming assignments\\
\indent - Deep neural networks for image and time-series data (\textit{e.g.} YOLO, VNN, LSTM, Attention Networks) \\
\vspace{-0.9em}
\phantom{.}\\
\phantom{.} \textbf{Board gaming and data science}\\
\indent - Own a library of about 50 board games and enjoy organizing weekly game nights\\
\indent - Scraped comments and ratings of more than 500 games on \textit{boardgamegeek.com}\\
\indent - Coded a collaborative filtering based recommendation algorithm in python using scraped ratings\\
\vspace{-0.9em}
\phantom{.}\\
\phantom{.} \textbf{Data Rescue Boston} \dotfill February 2017\\
\indent - Volunteer hackathon hosted at MIT for preserving data from U.S. government web pages\\
\indent - Developed web scraping scripts in python with selenium and requests to scrape pdfs and parse html/xml\\
\vspace{-2.25em}
\section*{Publications}
\vspace{-1.85em}
\noindent\rule{\textwidth}{0.4pt}
\vspace{-0.9em}
\phantom{.}\\
\noindent
- \textbf{W.E. Lewis} and P. Romatschke, ``Higher-harmonic collective modes of trapped gases from second-order hydrodynamics.'' New J. Phys. \textbf{19} 023042 (2017).\\
- H. Bantilan, J.T. Brewer, T. Ishii, \textbf{W.E. Lewis}, and P. Romatschke, ``String-theory-based predictions for nonhydrodynamic collective modes in strongly interacting Fermi gases." \textit{Phys. Rev. A} \textbf{94}, 033621 (2016).\\
- B.T. O'Callahan, \textbf{W.E. Lewis}, S. M\"{o}bius, J.C. Stanley, E.A. Muller, and M.B. Raschke, ``Broadband infrared vibrational spectroscopy using thermal blackbody radiation." 
\textit{ Opt. Exp.} \textbf{23}, 032063 (2015). \\
- \textbf{W.E. Lewis} and R. Vyas, ``Maxwell-Gaussian beams with cylindrical polarization."
\textit{J. Opt. Soc. Am. A} \textbf{31}, 1595 (2014). \\
- B.T. O'Callahan, \textbf{W.E. Lewis}, A.C. Jones, and M.B. Raschke, ``Spectral frustration and spatial coherence in thermal near-field spectroscopy." \textit{ Phys. Rev. B} \textbf{89}, 245446  (2014). \\
\vspace{-2.25em}
\section*{Presentations}
\vspace{-1.85em}
\noindent\rule{\textwidth}{0.4pt}
\vspace{-0.9em}
\phantom{.}\\
\noindent
- \textbf{W.E. Lewis} and P. Romatschke, ``Collective modes of a trapped gas from second-order hydrodynamics.'' APS March Meeting, New Orleans, LA (2017).\\
- \textbf{W.E. Lewis}, B.T. O'Callahan, A. Jones, and M.B. Raschke, ``Spectral frustration and spatial coherence in thermal near-field spectroscopy
.'' Near-Field Optics, Snowbird, UT (2014).\\�
- \textbf{W.E. Lewis}, B.T. O'Callahan, A. Jones, and M.B. Raschke, ``Thermal infrared near-field spectroscopy: spectroscopy of the resonant enhancement of the local electromagnetic density of states.'' APS Four Corners, Denver, CO (2014).\\�
- \textbf{W.E. Lewis}, R. Vyas, and S. Singh, ``Cylindrically polarized solutions of paraxial Maxwell's equations.'' Frontiers in Optics, San Jose, CA (2011). \\
- Invited: \textbf{W.E. Lewis}, ``Cylindrical vector beams: ensuring Maxwell's equations hold.'' University of Arkansas Sigma Pi Sigma Induction Ceremony, Fayetteville, AR (2011).\\
- \textbf{W.E. Lewis}, R. Vyas, S. Singh,``Generating cylindrically polarized Laguerre-Gauss beams interferometrically: An experiment proposal.''  INBRE conference, Fayetteville, AR (2011) (poster).\\
- \textbf{W.E. Lewis}, R.Vyas, and S. Singh, ``Bessel-Gauss beams with radial and azimuthal polarization.'' Frontiers in Optics, Rochester, NY (2010) (poster).\\
- \textbf{W.E. Lewis}, R.Vyas, and S. Singh, ``Bessel-Gauss beams with radial and azimuthal polarization.'' INBRE conference, Fayetteville, AR (2010) (poster).\\
\vspace{-1.75em}
%\section*{Awards and Honors}
%\vspace{-1.85em}
%\noindent\rule{\textwidth}{0.4pt}
%\vspace{-0.9em}
%\phantom{.}\\
%\noindent
%- University of Colorado Graduate School Domestic Travel Grant \dotfill 2017\\
%- National Barry M. Goldwater Scholarship \dotfill 2011-2012 \\
%- University of Arkansas Dept. of Physics George D. Lingelbach Outstanding Senior Award \dotfill 2011-2012\\
%- University of Arkansas Ellen and Gary Cleveland Scholarship \dotfill 2011-2012 \\
%- National Society of Physics Students Outstanding Leadership Scholarship \dotfill 2011-2012\\
%- University of Arkansas Dept. of Mathematics Lawrence Jesser Toll, Jr. Endowed Scholarship\dotfill 2011-2012\\
%- University of Arkansas Dept. of Physics Robert D. Maurer Fellowship \dotfill 2011-2012\\
%- Inducted into the National Honors Society Phi Beta Kappa \dotfill 2011\\
%- Inducted into the Physics Honors Society Sigma Pi Sigma \dotfill 2011\\
%- Arkansas Dept. of Higher Ed. Student Undergraduate Research Fellowship \dotfill 2011\\
%- University of Arkansas Chancellor's Scholarship\dotfill 2007-2011\\
%- Arkansas Dept. of Higher Ed. Governor's Distinguished Scholarship \dotfill 2007-2011\\
%\vspace{-1.75em}
%\section*{Teaching Experience}
%\vspace{-1.85em}
%\noindent\rule{\textwidth}{0.4pt}
%\vspace{-0.75em}
%\phantom{.}\\
%\phantom{.} \textbf{Teaching Assistant: Depts. of Physics and Chemistry University of Colorado} \dotfill 2012 - 2017\\
%\indent - Conducted introductory physics chemistry lab courses with class sizes ranging from about 10-30 students.\\ 
%\indent - Had contact with about 500 students in the course of 5 years of teaching.\\
%\indent - Lectured an advanced undergraduate optics course with about 20 students three times.\\
%\phantom{.} \textbf{Learning Assistant and Tutor: University of Arkansas} \dotfill 2008 - 2012\\
%\indent - Tutored mathematics and physics courses ranging from introductory to advanced level.\\ 
%\indent - Had contact with about 100 students in the course of 4 years of tutoring.\\
%\indent - 1 semester undergraduate learning assistant in an introductory physics lab of about 20 students.\\


\end{document}
